\documentclass[]{article}
\usepackage{lmodern}
\usepackage{amssymb,amsmath}
\usepackage{ifxetex,ifluatex}
\usepackage{fixltx2e} % provides \textsubscript
\ifnum 0\ifxetex 1\fi\ifluatex 1\fi=0 % if pdftex
  \usepackage[T1]{fontenc}
  \usepackage[utf8]{inputenc}
\else % if luatex or xelatex
  \ifxetex
    \usepackage{mathspec}
  \else
    \usepackage{fontspec}
  \fi
  \defaultfontfeatures{Ligatures=TeX,Scale=MatchLowercase}
\fi
% use upquote if available, for straight quotes in verbatim environments
\IfFileExists{upquote.sty}{\usepackage{upquote}}{}
% use microtype if available
\IfFileExists{microtype.sty}{%
\usepackage{microtype}
\UseMicrotypeSet[protrusion]{basicmath} % disable protrusion for tt fonts
}{}
\usepackage[margin=1in]{geometry}
\usepackage{hyperref}
\hypersetup{unicode=true,
            pdftitle={Expectations bias moral evaluations},
            pdfauthor={Derek Powell and Zachary Horne},
            pdfborder={0 0 0},
            breaklinks=true}
\urlstyle{same}  % don't use monospace font for urls
\usepackage{graphicx,grffile}
\makeatletter
\def\maxwidth{\ifdim\Gin@nat@width>\linewidth\linewidth\else\Gin@nat@width\fi}
\def\maxheight{\ifdim\Gin@nat@height>\textheight\textheight\else\Gin@nat@height\fi}
\makeatother
% Scale images if necessary, so that they will not overflow the page
% margins by default, and it is still possible to overwrite the defaults
% using explicit options in \includegraphics[width, height, ...]{}
\setkeys{Gin}{width=\maxwidth,height=\maxheight,keepaspectratio}
\IfFileExists{parskip.sty}{%
\usepackage{parskip}
}{% else
\setlength{\parindent}{0pt}
\setlength{\parskip}{6pt plus 2pt minus 1pt}
}
\setlength{\emergencystretch}{3em}  % prevent overfull lines
\providecommand{\tightlist}{%
  \setlength{\itemsep}{0pt}\setlength{\parskip}{0pt}}
\setcounter{secnumdepth}{0}
% Redefines (sub)paragraphs to behave more like sections
\ifx\paragraph\undefined\else
\let\oldparagraph\paragraph
\renewcommand{\paragraph}[1]{\oldparagraph{#1}\mbox{}}
\fi
\ifx\subparagraph\undefined\else
\let\oldsubparagraph\subparagraph
\renewcommand{\subparagraph}[1]{\oldsubparagraph{#1}\mbox{}}
\fi

%%% Use protect on footnotes to avoid problems with footnotes in titles
\let\rmarkdownfootnote\footnote%
\def\footnote{\protect\rmarkdownfootnote}

%%% Change title format to be more compact
\usepackage{titling}

% Create subtitle command for use in maketitle
\newcommand{\subtitle}[1]{
  \posttitle{
    \begin{center}\large#1\end{center}
    }
}

\setlength{\droptitle}{-2em}

  \title{Expectations bias moral evaluations}
    \pretitle{\vspace{\droptitle}\centering\huge}
  \posttitle{\par}
    \author{Derek Powell and Zachary Horne}
    \preauthor{\centering\large\emph}
  \postauthor{\par}
    \date{}
    \predate{}\postdate{}
  

\begin{document}
\maketitle

\section{Introduction}\label{introduction}

On the evening of November 13th, 2015, a terrorist attack in Paris left
130 people dead and injured over 300 more. In the aftermath, millions
took to Twitter to express their shock, horror, and outrage at this
tragedy under hashtags like \texttt{\#parisattacks} and
\texttt{\#jesuisparis}. Yet, most of those mourning had little to say 15
hours earlier, when another tragic attack killed at least 43 people in
Beirut. Several factors are surely at play in these different reactions
(construal-level theory, {\textbf{???}}; e.g.~group affiliations,
Brewer, 1999), yet one potentially fundamental factor has gone
unmentioned: the fact that the Paris attack was more surprising than the
attack in Beirut. In contrast to France, Lebanon had experienced dozens
of terrorist bombings and attacks in recent years. Consequently, Beirut
may seem to many like the sort of place where ``these things happen,''
whereas Paris is perceived as being stable and safe.

We can see in everyday experience that people's evaluations of events
often depend on their expectations about those events. There is often
disappointment when events fail to meet expectations, and there is a
special thrill to having one's expectations exceeded. Anecdotally, these
forces seem to drive people's tendencies to root for the underdog, hold
surprise parties, and foreshadow bad news to ease its delivery (Bell,
1985). Indeed, laboratory studies suggest that expectations play an
important role in people's evaluations of the utility of an event. For
instance, Mellers and colleagues (1997) found that expectations
influenced affective reactions during a gambling task: Given a gamble
with a 10\% chance to win \$30 and 90\% chance to win \$0, participants
felt little disappointment with the \$0 outcome, but were considerably
more excited when they won \$30. Conversely, given a gamble with a 90\%
chance of winning \$60 and 10\% chance of winning \$0, participants were
disappointed with the \$0 outcome and showed more muted enjoyment of the
\$60 outcome. In fact, in gambles similar to these, Mellers and
colleagues (1997) found that people were happier with the smaller
unexpected gain than with the larger but more expected gain (also see
Shepperd \& Mcnulty, 2002).

Just as expectations affect the utility of simple gambles, we suspect
they are equally likely to shape how people react to morally harmful
events, such as acts of terrorism (Waldmann Cite and Rai and Fiske cite
--domain general peeps; us also). However, unlike in the context of
gambles, in these contexts the effects of expectations on evaluations
may have harmful consequences. When events are shocking, we predict that
people will perceive them as more severe and consequently be roused to
action. In contrast, when events harm victims who are generally
considered to be at greater risk--the poor, sick, or those living in
unstable regions of the world, we predicted that reactions may be more
muted. However, prior research has yet to articulate whether these same
effects emerge in the moral domain and the implications of this
possibility. For instance, it is unclear if these same mechanisms can
help account for the divergent reactions people have to terrorism in
France and Beirut. Further, people's moral evaluations are in fact
impacted by expectations what is their source and what does that tell us
about how to overcome the biases they may impose?

A number of researchers have sought to develop theories of
disappointment--the psychological reactions that result when experiences
fail to meet expectations--and its role in evaluation and
decision-making (e.g., Bell, 1985; Gul, 1991; Loomes \& Sugden, 1986).
These researchers have generally argued that expectations modify
people's reactions to events, so that they are jointly determined by the
context-free utilities of options or events (e.g., economic utilities),
and the contextually-dependent disappointment that individuals
experience as a function of their expectations. Similarly, Mellers and
colleagues (1997) interpreted the divergence they observed between the
utility of an outcome (i.e., monetary gains or losses) and the felt
experience of that outcome as a result of post-decision affect.

Each of these theories has been rigorously defined and supported within
the small-world confines they were meant to capture. However, we seek to
craft a more general explanation of how and why expectations influence
evaluations and the range of contexts in which they should be expected
to do so. Several accounts of utility evaluation, including early
theories like Prospect Theory (Kahneman \& Tversky, 1979; Tversky \&
Kahneman, 1992), have emphasized the role of relative comparisons in
evaluations of utility. In a similar spirit, we argue that expectations
can set the reference points against which people compare future
outcomes. On this view, evaluating the utility of some event consists in
comparing the state of the world now that the event has occurred to the
state of the world just prior to the event. Further, we assume that our
knowledge of the world is uncertain, so that it is really just our
(probabilistic) expectations about what has happened, what is happening,
and what will happen in the future. Thus, the evaluation of an event
might be driven by what is learned about the state of the world as a
result of that event's occurrence.

The account outlined above provides an information-theoretic route to
understanding how expectations directly influence evaluation. A
fundamental insight of Information Theory is that the information
carried by an event is a function of its prior probability. This means
that low probability events carry more information than high probability
events (Shannon, 1948). Along similar lines, violations of expectations
have long been recognized as fundamental to associative and animal
learning models; a larger delta means people learn more (e.g., Rescorla
\& Wagner, 1972). Science progresses most when new findings impugn
widely-accepted theories or when a theory's extreme predictions are
validated. Likewise, people learn more about the state of the world when
their expectations are violated by shocking world events as compared to
when they are affirmed by less surprising events. For example, if a
bombing occurs in Paris -- an unexpected location -- rather than Lebanon
--a more expected location--we learn that the world is more dangerous
than we had previously believed.

A more formal treatment of these issues reveals the generality of these
conclusions. Formally, before some binary event \(x_i\) (which either
occurs or does not occur), an evaluator has a mental model of the
current state of affairs, S. This mental model includes expectations
about potential future events and uncertainty over current and past
events. Thus S represents a joint probability distribution over possible
states of affairs \(p(x_1, x_2, ..., x_n)\). The evaluator's subjective
utility function can assign an expected utility to S--representing how
positive or negatively the evaluator views the current state of affairs.
The occurrence of \(x_i\) brings about some new state of affairs, S', to
which the evaluator can again apply her utility function. The evaluation
of event \(x_i\) then consists in the comparison between these two
states of affairs, those before the event (reflecting the agents'
expectations), and the new state of affairs brought about by the event:

\[ V(x_i) = EU(S') -EU(S) \]

We let S be represented by a random vector X, whose members are any
number of discrete events \(x_1, x_2, ..., x_n\), where \(x_i\)
represents the event of interest so that \(x_i\) occurs in S with some
probability and occurs in S' with probability 1. That is, S represents
\(X\) and S' represents \(X|x_i=1\). We write this as:

\[
\begin{aligned}
V(x_i) &= EU(X|x_i) - EU(X) \\
&= \sum_{x \in X} U(x)p(x|x_i) - \sum_{x \in X} U(x)p(x)
\end{aligned}\]

Applying the chain rule of probability and some algebra (see appendix)
we eventually obtain:

\[V(x_i) = \big(1 - p(x_i) \big) \sum_{x \in X} U(x)p(x|x_i)\]

Thus, the value assigned to event \(x_i\) is proportional to the prior
probability of \(x_i\). Further, this holds true regardless of the
nature of the event in question and of the form of the utility function
over events.

In the context of our reaction to surprise parties, it might seem
desirable that expectations would influence our experience of events.
Much joy is tied to the psychological marriage between surprise and
evaluation. However, this basic cognitive process may have negative
moral repercussions. When harms are expected, these dynamics may lead
people to draw less extreme evaluations, feel less concern for victims,
and be less driven to help. To test this hypothesis, we examined whether
people's evaluations of morally harmful events are affected by their
expectations about those events. We asked people to compare pairs of
simple events where a victim suffered an identical harm, but where the
events differed in their prior probability. For each pair of events,
participants were asked to judge which event was more upsetting. Two
studies provided evidence for the predicted effect of expectations on
moral evaluations: people tended to view unexpected negative events as
more upsetting, even when the harm to victims was identical.

\section{General Methods}\label{general-methods}

Here we present two preregistered studies examining the role of
expectations in moral evaluations. Study 1 used a forced-choice task to
test the hypothesis that people would be more upset about moral outcomes
that were unexpected than expected. In Study 2, we tested of our
hypothesis using a more conservative judgment task to increase the
generalizability of our results.

\subsection{Materials}\label{materials}

In both studies, participants were presented with a series of trials
where they read brief (one sentence) descriptions of two different
events and were asked to indicate which of the two events seemed more
upsetting. In ``experimental'' trials, the two events were highly
similar, but differed in their prior probabilities: one event was more
expected and one more unexpected. (The perceived likelihood of a given
event was confirmed in prior norming studies). These prior expectations
were manipulated by changing the context in which the events occurred.
For example, participants considered the following stimulus:

\begin{itemize}
\item
  ``A 30 year old man in California dies in an earthquake''
  {[}Expected{]}
\item
  ``A 30 year old man in Oklahoma dies in an earthquake''
  {[}Unexpected{]}
\end{itemize}

In each event, the harm to the victim is the same (here, death) but one
event is more expected than the other, given the different likelihoods
of earthquakes occurring in California versus Oklahoma.

Each study contained between 16 experimental event-pairs that spanned a
variety of different events and contexts. All experimental materials for
these studies are available as supplemental online materials at {[}OSF
LINK{]}.

Both studies included ``equivalent'' filler trials. In these trials, the
two events differed in trivial contextual details that we did not expect
would affect participants' judgments. For example:

\begin{itemize}
\item
  ``A man in Connecticut starts a house fire.'' {[}Equally expected{]}
\item
  ``A man in New Hampshire starts a house fire.'' {[}Equally expected{]}
\end{itemize}

These filler trials were meant to prevent participants from becoming
explicitly aware of the structure of the experimental trials. Finally,
both studies included ``non-equivalent'' filler trials, the two events
differed substantially in the degree of harm suffered by a victim, so
that one event was expected to be seen as considerably more upsetting
than the other. For example:

\begin{itemize}
\item
  ``An 11-year-old child sets a doll on fire'' {[}Less severe{]}
\item
  ``A 12-year-old child sets a cat on fire'' {[}More severe{]}
\end{itemize}

These trials were included to allow participants a chance to use the
extremes of the response scale and to reduce any task demands that might
drive them to make artificially fine-grained distinctions between the
severity of events.

\subsection{Exclusions}\label{exclusions}

In both studies, we excluded participants who failed to incorrectly
answer attention-check questions. These questions asked participants to
enter a particular response to ensure that they were paying attention
and reading the items as they proceeded through the study. A final
question asked participants if they had paid attention and taken the
study seriously, encouraging them to be honest in their replies.

\subsection{Data Analysis}\label{data-analysis}

We analyzed our data by performing Bayesian estimation using the
probabilistic programming language Stan (Carpenter et al., 2017). We
tested our predictions by computing Bayes Factors (i.e.~BF01) on the
intercept term of our regression model using the hypothesis function in
the R package brms. Bayes Factors express the ratio of the probability
of data under the null hypothesis to the probability of the data under
an alternative hypothesis. Therefore, larger Bayes Factors indicate that
the data are more likely under the null hypothesis (e.g., that the
intercept is not different from zero) than the alternative hypothesis
(e.g., that the intercept is different from zero), and vice versa. Bayes
Factors can be influenced by prior choices (Gelman, Simpson, \&
Betancourt, 2017) so we also performed prior robustness checks to
confirm that our conclusions were unchanged under different prior
assumptions.

\section{Appendix}\label{appendix}

\[ \begin{aligned}
V(x_i) &= EU(X|x_i) - EU(X) \\
&=  \sum_{x \in X} U(x)p(x|x_i) - \sum_{x \in X} U(x)p(x) \\
&= \sum_{x \in X} U(x)p(x|x_i) - \sum_{x \in X} U(x)p(x|x_i)p(x_i) \\
&= \sum_{x \in X} U(x)p(x|x_i) - p(x_i)\sum_{x \in X} U(x)p(x|x_i) \\
&= \big(1 - p(x_i) \big) \sum_{x \in X} U(x)p(x|x_i) \\
\end{aligned}\]

The first step utilizes the Law of the Unconscious Statistician, the
next comes from the chain rule of probability, and the final two steps
are simple algebra.

\section{References}\label{references}

\setlength{\parindent}{-0.25in} \setlength{\leftskip}{0.125in} \noindent

\hypertarget{refs}{}
\hypertarget{ref-Bell1985}{}
Bell, D. (1985). Disappointment In Decision Making Under Uncertainty.
\emph{Operations Research}, \emph{33}(1), 1--27.
\url{http://doi.org/10.1287/opre.33.1.1}

\hypertarget{ref-Brewer1999}{}
Brewer, M. B. (1999). The psychology of prejudice: Ingroup love or
outgroup hate? \emph{Journal of Social Issues}, \emph{55}(3), 429--444.
\url{http://doi.org/10.1111/0022-4537.00126}

\hypertarget{ref-Carpenter2017}{}
Carpenter, B., Gelman, A., Hoffman, M. D., Lee, D., Goodrich, B.,
Betancourt, M., \ldots{} Riddell, A. (2017). Stan: A probabilistic
programming language. \emph{Journal of Statistical Software},
\emph{76}(1). \url{http://doi.org/10.18637/jss.v076.i01}

\hypertarget{ref-Gul1991}{}
Gul, F. (1991). A theory of disappointment aversion.
\emph{Econometrica}, \emph{59}(3), 667--686.
\url{http://doi.org/10.2307/2938223}

\hypertarget{ref-Kahneman1979}{}
Kahneman, D., \& Tversky, A. (1979). Prospect Theory: An Analysis of
Decision under Risk. \emph{Econometrica}, \emph{47}(2), 263--292.

\hypertarget{ref-Loomes1986}{}
Loomes, G., \& Sugden, R. (1986). Disappointment and Dynamic Consistency
in Choice under Uncertainty. \emph{Review of Economic Studies},
\emph{53}(2), 271--282. \url{http://doi.org/10.2307/2297651}

\hypertarget{ref-Mellers1997}{}
Mellers, B. a, Schwartz, a., Ho, K., \& Ritov, I. (1997). Decision
Affect Theory: Emotional Reactions to the Outcomes of Risky Options.
\emph{Psychological Science}, \emph{8}(6), 423--429.
\url{http://doi.org/10.1111/j.1467-9280.1997.tb00455.x}

\hypertarget{ref-Rescorla1972}{}
Rescorla, R. A., \& Wagner, A. R. (1972). A theory of Pavlovian
conditioning: Variations in the effectiveness of reinforcement and
nonreinforcement. In \emph{Classical conditioning ii current research
and theory} (Vol. 21, pp. 64--99).
\url{http://doi.org/10.1101/gr.110528.110}

\hypertarget{ref-Shannon1948}{}
Shannon, C. E. (1948). A Mathematical Theory of Communication.
\emph{Bell System Technical Journal}, \emph{27}(3), 379--423.
\url{http://doi.org/10.1002/j.1538-7305.1948.tb01338.x}

\hypertarget{ref-Shepperd2002}{}
Shepperd, J. a, \& Mcnulty, J. K. (2002). The affective consequences of
expected and unexpected outcomes. \emph{Psychological Science},
\emph{13}(1), 85--88. \url{http://doi.org/10.1111/1467-9280.00416}

\hypertarget{ref-Tversky1992}{}
Tversky, A., \& Kahneman, D. (1992). Advances in prospect theory:
Cumulative representation of uncertainty. \emph{Journal of Risk and
Uncertainty}, \emph{5}, 297--323.


\end{document}
